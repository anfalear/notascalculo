\documentclass[12pt, fleqn, leqno]{book}

% ==========================================
% CONFIGURACIÓN DE PÁGINA Y CODIFICACIÓN
% ==========================================
\usepackage[paperheight=13in,paperwidth=8.5in,
            left=1.5cm,right=1.5cm,
            top=2cm,bottom=2cm]{geometry}
\usepackage[utf8]{inputenc}
\usepackage[spanish]{babel}

% ==========================================
% PAQUETES MATEMÁTICOS
% ==========================================
\usepackage{amsmath,amsthm,amssymb,amsfonts}
\usepackage{mathrsfs}

% ==========================================
% PAQUETES GRÁFICOS Y FIGURAS
% ==========================================
\usepackage{graphicx}
\usepackage{xcolor}
\usepackage{float}
\usepackage{wrapfig}
\usepackage[rflt]{floatflt}

% ==========================================
% PAQUETES DE TABLAS
% ==========================================
\usepackage{array}
\usepackage{booktabs}
\usepackage{tabularx}
\usepackage{tabu}
\usepackage{colortbl}
\usepackage{multirow}
\usepackage{rotating}

% ==========================================
% PAQUETES TIKZ Y GRÁFICOS
% ==========================================
\usepackage{pgf,tikz,pgfplots}
\pgfplotsset{compat=1.18, width=10cm}
\usetikzlibrary{
    decorations.markings,
    arrows,
    babel,
    backgrounds,
    calc,
    intersections,
    positioning,
    graphs,
    graphdrawing,
    matrix
}
\usegdlibrary{trees, force}
\usepackage{circuitikz}

% ==========================================
% PAQUETES MISCELÁNEOS
% ==========================================
\usepackage{multicol}
\usepackage{enumerate}
\usepackage{setspace}
\usepackage{fancybox}
\usepackage{parskip}
\usepackage{changepage}
\usepackage{hyperref}
\usepackage{url}
\usepackage{listings}
\usepackage{nameref, cancel}



% ==========================================
% PAQUETE TCOLORBOX
% ==========================================
\usepackage[most]{tcolorbox}
\tcbuselibrary{listingsutf8,raster,skins,most,theorems,breakable}

% ==========================================
% COMANDOS MATEMÁTICOS PERSONALIZADOS
% ==========================================
\newcommand{\imp}{\rightarrow}
\newcommand{\bimp}{\leftrightarrow}
\newcommand{\y}{\wedge}
\newcommand{\un}{\vee}
\newcommand{\xun}{\vee}
\newcommand{\N}{{\mathbb N}}
\newcommand{\Z}{{\mathbb Z}}
\newcommand{\I}{{\mathbb I}}
\newcommand{\R}{{\mathbb R}}
\newcommand{\Q}{{\mathbb Q}}
\newcommand{\FF}{{\mathbb F}}
\newcommand{\abs}[1]{\left\vert#1\right\vert}


% ==========================================
% CONFIGURACIÓN DE TIEMPO
% ==========================================
\newcommand{\tiempo}{1.5~horas}

% ==========================================
% COLORES PERSONALIZADOS
% ==========================================
\definecolor{uiscolor}{RGB}{103, 185, 62}
\definecolor{uiscolorb}{RGB}{201, 149, 0}
\definecolor{rulecolor}{named}{uiscolor}

% ==========================================
% CAJA DE SOLUCIÓN PERSONALIZADA
% ==========================================
\newtcolorbox{myproof}{detach title,before upper={\tcbtitle\quad},colback=uiscolor!5!white,colframe=uiscolor!75!black,coltitle=red!85!black,title=\textbf{Solución:},enhanced, breakable,overlay broken = {
        \draw[line width=0.2mm, uiscolor!75!black, rounded corners]
        (frame.north west) rectangle (frame.south east);}}

% ==========================================
% ESTILOS PERSONALIZADOS DE CAPÍTULOS Y SECCIONES
% ==========================================
\usepackage[explicit]{titlesec}
\titleformat{\chapter}[display]
  {\normalfont\huge\bfseries}
  {}
  {1pt}
  {%
    \begin{tcolorbox}[
      enhanced,
      colback=uiscolor!5!white,
      boxrule=0.25cm,
      colframe=uiscolor!75!black,
      arc=0pt,
      outer arc=0pt,
      leftrule=0pt,
      rightrule=0pt,
      fontupper=\color{red!85!black}\centering\sffamily\bfseries\huge,
      enlarge left by=-1in-\hoffset-\oddsidemargin, 
      enlarge right by=-\paperwidth+1in+\hoffset+\oddsidemargin+\textwidth,
      width=\paperwidth, 
      left=1in+\hoffset+\oddsidemargin, 
      right=\paperwidth-1in-\hoffset-\oddsidemargin-\textwidth,
      top=0.7cm, 
      bottom=0.3cm,
      overlay={
        \node[fill=uiscolor!5!white,
          draw=uiscolor!75!black,
          line width=0.15cm,
          inner sep=0pt,
          text width=1.7cm,
          minimum height=1.7cm,
          align=center,
          font=\color{red!85!black}\sffamily\bfseries\fontsize{30}{36}\selectfont
        ] 
        (chapname)
        at ([xshift=-0.5\paperwidth]frame.north east)
        {\thechapter};
        \node[font=\color{red!85!black}\sffamily\bfseries,anchor=south,inner sep=2pt] at (chapname.north){\chaptertitlename};
      }
    ]
    #1
    \end{tcolorbox}%
  }
\titleformat{name=\chapter,numberless}[display]
  {\normalfont\huge\bfseries}
  {}
  {1pt}
  {%
    \begin{tcolorbox}[
      enhanced,
      colback=uiscolor!5!white,
      boxrule=0.25cm,
      colframe=uiscolor!75!black,
      arc=0pt,
      outer arc=0pt,
      remember as=title,
      leftrule=0pt,
      rightrule=0pt,
      fontupper=\color{red!85!black}\centering\sffamily\bfseries\huge,
      enlarge left by=-1in-\hoffset-\oddsidemargin, 
      enlarge right by=-\paperwidth+1in+\hoffset+\oddsidemargin+\textwidth,
      width=\paperwidth, 
      left=1in+\hoffset+\oddsidemargin, 
      right=\paperwidth-1in-\hoffset-\oddsidemargin-\textwidth,
      top=0.25cm, 
      bottom=0.25cm, 
    ]
    #1
    \end{tcolorbox}%
  }
\titlespacing*{\chapter}{0pt}{0pt}{1pt}

\titleformat{\section}[display]
  {\normalfont\bfseries}
  {}
  {1pt}
  {%
    \begin{tcolorbox}[
      enhanced,
      colback=uiscolor!5!white,
      boxrule=0.1cm,
      colframe=uiscolor!75!black,
      arc=0pt,
      outer arc=0pt,
      leftrule=0.0cm,
      rightrule=0.0cm,
      fontupper=\color{red!85!black}\centering\sffamily\bfseries\Large,
      enlarge left by=-1in-\hoffset-\oddsidemargin, 
      enlarge right by=-\paperwidth+1in+\hoffset+\oddsidemargin+\textwidth,
      width=\paperwidth, 
      left=1in+\hoffset+\oddsidemargin, 
      right=\paperwidth-1in-\hoffset-\oddsidemargin-\textwidth,
      top=0.7cm, 
      bottom=0.3cm,
      overlay={
        \node[fill=uiscolor!5!white,
          draw=uiscolor!75!black,
          line width=0.1cm,
          inner sep=0pt,
          text width=2cm,
          minimum height=1.4cm,
          align=center,
          font=\color{red!85!black}\sffamily\bfseries\fontsize{20}{20}\selectfont
        ] 
        (chapname)
        at ([xshift=-0.5\paperwidth]frame.north east)
        {\thesection};
      }
    ]
    #1
    \end{tcolorbox}%
  }
\titlespacing*{\section}{0pt}{0pt}{1pt}

\titleformat{\subsection}[display]
  {\normalfont\bfseries}
  {}
  {1pt}
  {%
    \begin{tcolorbox}[
      enhanced,
      colback=uiscolor!5!white,
      boxrule=0.1cm,
      colframe=uiscolor!75!black,
      arc=0pt,
      outer arc=0pt,
      leftrule=0.0cm,
      rightrule=0.0cm,
      fontupper=\color{red!85!black}\centering\sffamily\bfseries\large,
      enlarge left by=-1in-\hoffset-\oddsidemargin, 
      enlarge right by=-\paperwidth+1in+\hoffset+\oddsidemargin+\textwidth,
      width=\paperwidth, 
      left=1in+\hoffset+\oddsidemargin, 
      right=\paperwidth-1in-\hoffset-\oddsidemargin-\textwidth,
      top=0.7cm, 
      bottom=0.1cm,
      overlay={
        \node[fill=uiscolor!5!white,
          draw=uiscolor!75!black,
          line width=0.1cm,
          inner sep=0pt,
          text width=2cm,
          minimum height=1.5cm,
          align=center,
          font=\color{red!85!black}\sffamily\bfseries\fontsize{15}{15}\selectfont
        ] 
        (chapname)
        at ([xshift=-0.5\paperwidth]frame.north east)
        {\thesubsection};
      }
    ]
    #1
    \end{tcolorbox}%
  }
\titlespacing*{\subsection}{0pt}{0pt}{1pt}

\titleformat{\part}[display]
  {\normalfont\Huge\bfseries}
  {}
  {1pt}
  {%
    \begin{tcolorbox}[
      enhanced,
      colback=uiscolor!5!white,
      boxrule=0.25cm,
      colframe=uiscolor!75!black,
      arc=0pt,
      outer arc=0pt,
      leftrule=0pt,
      rightrule=0pt,
      fontupper=\color{red!85!black}\centering\sffamily\bfseries\huge,
      enlarge left by=-1in-\hoffset-\oddsidemargin, 
      enlarge right by=-\paperwidth+1in+\hoffset+\oddsidemargin+\textwidth,
      width=\paperwidth, 
      left=1in+\hoffset+\oddsidemargin, 
      right=\paperwidth-1in-\hoffset-\oddsidemargin-\textwidth,
      top=0.7cm, 
      bottom=0.3cm,
      overlay={
        \node[fill=uiscolor!5!white,
          draw=uiscolor!75!black,
          line width=0.15cm,
          inner sep=0pt,
          text width=1.7cm,
          minimum height=1.7cm,
          align=center,
          font=\color{red!85!black}\sffamily\bfseries\fontsize{30}{36}\selectfont
        ] 
        (chapname)
        at ([xshift=-0.5\paperwidth]frame.north east)
        {\thepart};
        \node[font=\color{red!85!black}\sffamily\bfseries,anchor=south,inner sep=2pt] at (chapname.north){\partname};
      }
    ]
    #1
    \end{tcolorbox}%
  }

% ==========================================
% CONFIGURACIÓN DE REGLAS COLOREADAS
% ==========================================
\makeatletter
\let\old@rule\@rule
\def\@rule[#1]#2#3{\textcolor{rulecolor}{\old@rule[#1]{#2}{#3}}}
\makeatother

% ==========================================
% CONFIGURACIÓN DE TEOREMAS
% ==========================================
\usepackage{thmtools}

\theoremstyle{plain}
\newtheorem{theorem}{Teorema}[chapter]
\newtheorem{axm}[theorem]{Axioma}
\newtheorem{coro}[theorem]{Corolario} 
\newtheorem{lemma}[theorem]{Lema} 
\newtheorem{prop}[theorem]{Proposición} 
\newtheorem{affir}[theorem]{Afirmación} 
\newtheorem{rem}[theorem]{Observación} 
\newtheorem{wow}[theorem]{Para notar!!!} 
\newtheorem{para}[theorem]{Detente!!!} 

\theoremstyle{definition}
\newtheorem{example}[theorem]{Ejemplo} 
\newtheorem{prob}[theorem]{Problema} 
\newtheorem{definition}[theorem]{Definición} 
\newtheorem{pasos}[theorem]{Construcción} 
\newtheorem{appt}[theorem]{Aplicación}

% ==========================================
% CONFIGURACIÓN MATEMÁTICA
% ==========================================
\everymath{\displaystyle}

% ==========================================
% CONFIGURACIÓN DE TABLA DE CONTENIDOS
% ==========================================
\usepackage{tocloft}
\usepackage{titletoc}
\addto\captionsspanish{\renewcommand{\listtheoremname}{Glosario}}
\addto\captionsspanish{\renewcommand{\contentsname}{Contenido}}

% ==========================================
% CONFIGURACIÓN DE ENCABEZADOS Y PIES DE PÁGINA
% ==========================================
\usepackage{fancyhdr}
\pagestyle{fancy}
\fancyhead[LE,RO]{\sffamily\bfseries\thepage}
\fancyhead[LO]{\sffamily\bfseries \rightmark}
\fancyhead[RE]{\sffamily\bfseries\leftmark}
\fancyfoot[CE]{\sffamily\bfseries\textbf{Problemario de Cálculo}}
\fancyfoot[CO]{\sffamily\bfseries\textbf{Andrés Fabián Leal Archila}}
\renewcommand{\headrulewidth}{0.4pt}
\renewcommand{\footrulewidth}{0.4pt}
\begin{document} 
%\sf
	\setlength{\headheight}{15pt}
\begin{titlepage}
  \centering
  \vspace*{2cm}
  {\LARGE Problemario de Cálculo \par}
  \vspace{2cm}
  {\large Andrés Fabián Leal Archila\\ \url{anfalear@correo.uis.edu.co}\\ Escuela de Matemáticas \\ Universidad Industrial de Santander \\ \url{aleal214@unab.edu.co}\\ Departamento de Ciencias Básicas \\ Universidad Autónoma de Bucaramanga\par} 
  \vfill
 {\large Última actualización: \today \par}
 
  {\rule[0.1mm]{\textwidth}{0.5mm}
  \begin{flushright}
    \emph{``Si he visto más, es poniéndome sobre los hombros de Gigantes"}\\
    \textbf{\textit{Isaac Newton}.}
  \end{flushright}
  {\rule[0.1mm]{\textwidth}{0.5mm}}}
  \vspace*{2cm}
 
\end{titlepage}

\chapter*{Prefacio a la primera edición}
\addcontentsline{toc}{chapter}{Prefacio a la primera edición}

Este documento proporciona preparación para exámenes, talleres y quices de Cálculo I, II y III, presentando problemas prácticos y sus soluciones extraídas de diversas fuentes bibliográficas.

El documento inicia con las políticas del curso, contenidos, evaluación y contacto. Cada capítulo incluye secciones con resumen teórico, problemas resueltos y ejercicios de práctica. Al final se proponen rutinas de programación en Python 3.x, Wolfram Alpha, \url{www.matrixcalc.org/es} y GeoGebra.

Este material está en construcción continua, por lo que algunos ejercicios pueden carecer de solución o presentar errores que se corregirán progresivamente.

Espero que este documento contribuya a su proceso de aprendizaje.

\begin{flushright}
Cordialmente,\\
Andrés Fabián Leal Archila\\
Docente Escuela de Matemáticas\\
Universidad Industrial de Santander\\
Correo: \url{anfalear@correo.uis.edu.co}\\
Docente Departamento de Ciencias Básicas\\
Universidad Autónoma de Bucaramanga\\
Correo: \url{aleal214@unab.edu.co}
\end{flushright}


\newpage
\tableofcontents
\newpage
\listoftheorems [numwidth=3.5em]
\newpage
\listoffigures 

\chapter{Políticas del curso}

\noindent\fbox{\parbox{\textwidth}{\textit{
La información aquí presentada no se modificará por ninguna de las partes a menos que haya un consenso entre estas, el cual debe ser oportunamente informado y reportado por escrito en este documento. Ante cualquier instancia, este documento será soporte para las acciones a las que se de lugar, por lo cual, tanto el profesor como el estudiante lo aceptan y se someten a lo que esté aquí escrito. 
}}}

Esta política se rige de acuerdo a lo dispuesto en los Reglamentos Académicos de Pregrado de cada institución (R.A.P.) disponibles en las web institucionales.

Los cursos de Álgebra Lineal tendrán su Aula Virtual, donde se informarán horarios de atención, distribución de exámenes, demás aspectos particulares, calificaciones y distribuciones de porcentajes. La nota mínima de aprobación es 3.0.

\section{Contenido del curso} 

La cobertura total del contenido dependerá del progreso del semestre; es responsabilidad del estudiante mantenerse al día. Los temas evaluados en cada corte se proporcionarán oportunamente. Los temas marcados con asterisco (*) son opcionales según el progreso del semestre.

\begin{enumerate}
\item \textbf{Funciones}
\begin{enumerate}[$a)$]
\item Concepto y representación de funciones.
\item Clasificación y gráficas de funciones:
    \begin{itemize}
        \item Algebraicas: polinómicas, racionales e irracionales.
        \item Trascendentes: exponenciales, logarítmicas, trigonométricas, trigonométricas inversas e hiperbólicas.
        \item Otras funciones: valor absoluto y parte entera.
    \end{itemize}
\item Operaciones con funciones:
    \begin{itemize}
        \item Álgebra de funciones.
        \item Composición de funciones.
        \item Función inversa.
    \end{itemize}
\item Representación de funciones como modelos matemáticos.
\end{enumerate}

\item \textbf{Límites y continuidad de funciones}
\begin{enumerate}[$a)$]
\item Concepto de límite de una función.
\item Límites laterales, infinitos y especiales.
\item Teoremas de límites y cálculo de límites utilizando teoremas.
\item Límites de funciones trigonométricas.
\item Continuidad de funciones.
\end{enumerate}

\item \textbf{Derivada}
\begin{enumerate}[$a)$]
\item Interpretación geométrica de la derivada.
\item Velocidad promedio y velocidad instantánea.
\item Concepto y reglas para el cálculo de la derivada (incluyendo la regla de la cadena).
\item Derivadas de:
    \begin{itemize}
        \item Funciones algebraicas (teoremas y cálculo).
        \item Funciones trigonométricas e inversas.
        \item Funciones exponenciales y logarítmicas.
        \item Funciones hiperbólicas.
    \end{itemize}
\item Derivadas de orden superior.
\item Derivación implícita y función inversa.
\end{enumerate}

\item \textbf{Aplicaciones de la derivada}
\begin{enumerate}[$a)$]
\item Tasas de cambio relacionadas.
\item Diferenciales y aproximaciones.
\item Regla de L’hôpital.
\item Trazado de gráficas:
    \begin{itemize}
        \item Máximos y mínimos.
        \item Crecimiento, decrecimiento y concavidades.
        \item Puntos de inflexión y asíntotas.
    \end{itemize}
\item Teorema del valor medio.
\item Problemas de optimización.
\end{enumerate}

\item \textbf{Introducción a las integrales}
\begin{enumerate}[$a)$]
\item Concepto de antiderivadas o integración indefinida.
\item Técnicas de integración:
    \begin{itemize}
        \item Sustitución simple.
        \item Por partes.
        \item De funciones racionales mediante completación de cuadrados o fracciones parciales.
        \item Con productos y potencias de funciones trigonométricas.
        \item Sustitución trigonométrica.
    \end{itemize}
\item Integrales impropias.
\end{enumerate}

\item \textbf{La integral definida}
\begin{enumerate}[$a)$]
\item Concepto y propiedades de la integral definida.
\item El teorema fundamental del cálculo.
\item Cálculo de integrales definidas.
\item Integración numérica.
\end{enumerate}

\item \textbf{Aplicaciones de la integral}
\begin{enumerate}[$a)$]
\item Cálculo de áreas de regiones planas.
\item Volumen de sólidos de revolución (capas, discos, arandelas y cascarones).
\item Longitud de arco.
\item Áreas de superficies de revolución.
\item Centro de masa, centro de inercia y trabajo.
\end{enumerate}

\item \textbf{Ecuaciones paramétricas y coordenadas polares}
\begin{enumerate}[$a)$]
\item Representación paramétrica de curvas en el plano.
\item Sistema de coordenadas polares.
\item Gráfica de ecuaciones polares.
\item Cálculo de áreas y longitud en coordenadas polares.
\end{enumerate}
\end{enumerate}


\textbf{Notas aclaratorias:}
\begin{itemize}
\item No hay número específico de quices o trabajos; si son varios se promedian.
\item Las calificaciones se publican en la plataforma universitaria con notificación por correo. Si el estudiante no puede visualizar su nota, debe manifestarlo por correo al profesor.
\item El profesor acordará con los estudiantes la forma de revisar trabajos y exámenes.
\end{itemize}

\section{Políticas de clase}

\begin{itemize}
\item Los instrumentos electrónicos requieren aprobación del profesor.

\item Quices y exámenes sin apuntes, libros ni aparatos electrónicos (salvo flexibilización informada). El estudiante debe leer las reglas de cada evaluación.

\item El trabajo debe ser autónomo o en grupo designado. \textbf{Ofrecer} o \textbf{aceptar} soluciones ajenas constituye \textbf{plagio}, penalizado según el R.A.P. Incluye omisión de fuentes e uso no citado de IA.

\item \textbf{No se reciben trabajos fuera de fecha}, excepto con excusa según R.A.P.

\item \textbf{Asistencia obligatoria} con control por parte del profesor.

\item En exámenes escritos se esperan 15 minutos tras el inicio para ingresar; después se requiere supletorio según R.A.P.

\item \textbf{Casos de emergencia:} El profesor informará evacuaciones necesarias. No se responsabiliza por decisiones estudiantiles post-evacuación.
\end{itemize}
\chapter{Preliminares}

En esta sección introductoria se abordarán algunas propiedades fundamentales de los números reales, cuyo estudio resulta esencial para la comprensión rigurosa de los principios del cálculo. A partir del análisis del concepto de número real, del plano cartesiano y de otras nociones preliminares, se pretende establecer una base conceptual sólida que facilite la transición hacia el estudio formal del cálculo diferencial e integral, estructurado desde la noción de función.

\section{Números reales}

Uno de los conceptos más fundamentales en las matemáticas es el de número, cuya noción se remonta a la antigüedad y cuya generalización se ha desarrollado a lo largo de la historia. La estructura numérica básica se origina en el conjunto de los \textit{números naturales}, que posteriormente se amplió al incorporar los números negativos, dando lugar a los \textit{números enteros}. Más adelante, al considerar la posibilidad de expresar un número como una fracción de la forma $\dfrac{a}{b}$, donde $a$ y $b$ son enteros con $b \neq 0$, surge el conjunto de los \textit{números racionales}, los cuales pueden representarse mediante expresiones decimales finitas o periódicas infinitas. 

Desde la época de los pitagóricos, se reconoce la existencia de los \textit{números irracionales}, aquellos que no pueden expresarse como cociente de dos números enteros; un ejemplo clásico es $\sqrt{2}$. La unión de los números racionales e irracionales da origen al conjunto de los \textbf{números reales}, que constituirá el objeto principal de estudio en esta sección.

El descubrimiento de los números irracionales, atribuido a la escuela pitagórica en la antigua Grecia (siglo V a.\,C.), representó una crisis intelectual para la concepción aritmética de la época, pues reveló la existencia de magnitudes inconmensurables. Este hallazgo marcó un punto de inflexión en el pensamiento matemático, impulsando el desarrollo posterior de la teoría de los números y la formalización del concepto de continuidad numérica.

A continuación se presentan algunas propiedades fundamentales de los números reales, cuya comprensión resulta esencial para el estudio riguroso del análisis matemático.

\subsection{Propiedades de los números reales}

\begin{theorem}[Propiedad de tricotomía]
El conjunto de los números reales forma un cuerpo ordenado. En particular, dados $x$ y $y$ números reales, existe una y solo una de las siguientes relaciones: $x < y$, $x = y$ o $x > y.$
\end{theorem}

Los números reales pueden representarse sobre la \textbf{recta numérica infinita} (o \textit{eje real}) de acuerdo con la siguiente convención (figura \ref{rectanumerica}): se define un punto $O$, denominado \textit{origen}, que representa al número $0$; una flecha orientada hacia la derecha indica la dirección positiva, mientras que una flecha orientada hacia la izquierda representa los valores negativos. 

Por ejemplo, si un número $x_1$ es positivo, se ubica en la recta como un punto a la derecha del origen, a una distancia proporcional a su valor. De manera análoga, si un número $x_2$ es negativo, se representa como un punto a la izquierda del origen, también a una distancia proporcional a $|x_2|$. 

\begin{figure}[H] \label{rectanumerica}



\begin{tikzpicture}[>=stealth, scale=1.1]
  % Eje real
  \draw[->] (-4,0) -- (4,0) node[below right] {$x$};
  
  % Origen
  \draw[fill] (0,0) circle (1.5pt) node[below] {$0$};
  
  % Marca y etiqueta de un número positivo x_1
  \draw[fill] (2,0) circle (1.5pt) node[above] {$x_1>0$};
  
  % Marca y etiqueta de un número negativo x_2
  \draw[fill] (-2,0) circle (1.5pt) node[above] {$x_2<0$};
  
  % Indicaciones de direcciones positiva y negativa
  \node[below] at (3.6,0) {dirección positiva};
  \node[below] at (-3.6,0) {dirección negativa};
\end{tikzpicture}
\caption{Recta numérica o eje real}

\end{figure}

Existe así una correspondencia biunívoca entre los números reales y los puntos de la recta numérica: a cada número real le corresponde un punto único, y a cada punto, un número real. Además, debido a una propiedad denominada \textbf{completitud de los números reales} —cuyo tratamiento formal excede los alcances de este curso—, se cumple que entre dos números reales cualesquiera siempre es posible encontrar otros números, sean racionales o irracionales. Esta idea se resume en el siguiente teorema clásico.

\begin{theorem}
Para todo número irracional $\alpha$, existen infinitos números racionales que lo aproximan con cualquier grado deseado de precisión.
\begin{proof}
Sea $\alpha > 0$ un número irracional. Queremos calcular una aproximación con un error menor que $\dfrac{1}{n}$. Note que, para cualquier $\alpha$, existe un número entero $N$ tal que $N < \alpha < N + 1$. Si se divide el segmento comprendido entre $N$ y $N + 1$ en $n$ partes iguales, entonces $\alpha$ se encontrará entre $N + \dfrac{m}{n}$ y $N + \dfrac{m + 1}{n}$ para algún entero $m$. La diferencia entre ambos extremos es $\dfrac{1}{n}$, de modo que se puede alcanzar cualquier precisión deseada disminuyendo el valor de $\dfrac{1}{n}$.
\end{proof}
\end{theorem}

\begin{example}
El número irracional $\sqrt{2}$ puede aproximarse de la siguiente manera:
\begin{align*}
1.4 &< \sqrt{2} < 1.5,\\
1.41 &< \sqrt{2} < 1.42,\\
1.414 &< \sqrt{2} < 1.415,
\end{align*}
según precisiones de $\dfrac{1}{10}$, $\dfrac{1}{100}$ y $\dfrac{1}{1000}$, respectivamente.
\end{example}

\subsection{Valor absoluto de los números reales}

\begin{definition}[Valor absoluto]
El \textbf{valor absoluto} (o módulo) de un número real $x$, denotado por $|x|$, se define como:
\begin{align*}
|x| &= x & \text{si } x \geq 0,\\
|x| &= -x & \text{si } x < 0.
\end{align*}
Por esta definición, $|x|$ representa siempre un número real no negativo ($|x| \geq 0$) y mide la distancia de $x$ al origen en la recta numérica.
\end{definition}

\begin{example}
Para $x = 2$, se tiene $|2| = 2 \geq 0$. Para $x = -3$, se verifica $|-3| = 3 > 0$. En ambos casos, el valor absoluto produce un número no negativo.
\end{example}

A continuación se enuncian las propiedades fundamentales del valor absoluto:

\begin{theorem}[Propiedades del valor absoluto]
Dados $x, y \in \mathbb{R}$, se cumplen las siguientes propiedades:
\begin{enumerate}
\item $|x + y| \leq |x| + |y|$ (Desigualdad triangular).
\item $|x - y| \geq \bigl||x| - |y|\bigr|$ (Segunda desigualdad triangular).
\item $|xy| = |x| \cdot |y|$ para todo $x, y \in \mathbb{R}$.
\item $\left|\dfrac{x}{y}\right| = \dfrac{|x|}{|y|}$ siempre que $y \neq 0$.
\end{enumerate}
\end{theorem}

\begin{proof}


\begin{enumerate}
\item \textbf{Caso $1$:} $x + y \geq 0$. Entonces $|x + y| = x + y \leq |x| + |y|$, pues $|x| \geq x$ y $|y| \geq y$.

\textbf{Caso $2$:} $x + y < 0$. Entonces $|x + y| = -(x + y) = (-x) + (-y) \leq |x| + |y|$, ya que $|x| \geq -x$ y $|y| \geq -y$.

\item Sea $z = x - y$. Entonces $x = y + z$, y por la desigualdad triangular:
\[
|x| = |y + z| \leq |y| + |z| = |y| + |x - y|.
\]
Restando $|y|$ de ambos lados: $
|x| - |y| \leq |x - y|.$ 

Dado que la desigualdad es simétrica, también se tiene $|y| - |x| \leq |x - y|$, lo que equivale a:
\[
|x - y| \geq \bigl||x| - |y|\bigr|.
\]

Las propiedades $3$ y $4$ se siguen directamente de la definición del valor absoluto.
\end{enumerate}
\end{proof}



\subsection{Magnitudes variables y constantes e intervalos de variación}

Las \textbf{magnitudes variables}, tales como el tiempo, longitud, área, volumen, masa, velocidad y temperatura, obtienen sus valores mediante el hábito de la medición y se presentan en sucesión constante. Para el caso del movimiento uniforme, podemos observar el desplazamiento de un punto a lo largo del tiempo.

Por contraste, las \textbf{magnitudes constantes} son aquellas que no sufren variación; una magnitud variable cuyo valor permanece invariable se denomina constante. La distinción entre magnitudes variables y constantes resulta fundamental en el análisis matemático. De esto, por ejemplo $\pi\approx 3.14159$ es una constante mientras $t$ vista como el tiempo es variable.

Una magnitud variable puede tomar diferentes valores numéricos. Se considera el problema de la temperatura del agua al calentarla en condiciones normales, variará desde $15^\circ$C hasta el punto de ebullición: $100^\circ$C.

\begin{figure}[H]
\centering
\begin{tikzpicture}[scale=1.5]
  % Círculo unitario
  \draw[thick] (0,0) circle (1cm);
  % Ejes
  \draw[->] (-1.3,0) -- (1.3,0) node[right] {$x$};
  \draw[->] (0,-1.3) -- (0,1.3) node[above] {$y$};
  % Puntos cardinales
  \fill (-1,0) circle (1.5pt) node[below left] {$(-1,0)$};
  \fill (1,0) circle (1.5pt) node[below right] {$(1,0)$};
  \fill (0,1) circle (1.5pt) node[above] {$(0,1)$};
  % Punto genérico (x,y) en el primer cuadrante
  \fill[red] (0.707,0.707) circle (2pt) node[above right] {$(x,y)$};
  % Ángulo theta desde origen al punto (x,y)
  \draw[blue, thick] (0,0) -- (0.707,0.707) node[midway, sloped, above, font=\small] {$\theta$};
  % Arco para enfatizar el ángulo
  \draw[blue, dashed] (0.707,0) arc (0:45:0.707cm);
\end{tikzpicture}
\caption{Representación trigonométrica en el círculo unitario: $x = \cos\theta$, $y = \sin\theta$, donde $(x,y)$ satisface $x^2 + y^2 = 1$}
\label{fig:circulo}
\end{figure}


La variable $z = \cos\theta$ puede tomar todos los valores comprendidos entre $-1$ y $+1$. El conjunto de todos los valores posibles que puede tomar la variable independiente $\theta$ se denomina \textbf{campo de variación} de la variable independiente, el cual determina el conjunto de valores posibles de la función dependiente. 

Por ejemplo, en la función $z = \cos\theta$, el \textbf{campo de variación} de $z$ es el intervalo cerrado $[-1,1]$. Posteriormente, este concepto se denominará formalmente \textit{rango} (o \textit{imagen}) de una función.

Observe que hemos introducido la notación de intervalo $[-1,1]$. Esta notación se define de la siguiente manera:


\begin{definition}[Intervalos]
Un \textbf{intervalo} es el conjunto de valores numéricos $x \in \mathbb{R}$ comprendidos entre dos números reales dados $a$ y $b$ ($a < b$). Los intervalos se clasifican según los extremos incluidos:

\begin{itemize}
\item \textbf{Intervalo abierto}: $(a,b) = \{ x \in \mathbb{R} \mid a < x < b \}$. \textit{No se incluyen los extremos $a$ ni $b$.}
\item \textbf{Intervalo cerrado}: $[a,b] = \{ x \in \mathbb{R} \mid a \leq x \leq b \}$. \textit{Se incluyen ambos extremos.}
\item \textbf{Intervalos semicerrados (o semiabiertos)}:
  \begin{itemize}
  \item $(a,b] = \{ x \in \mathbb{R} \mid a < x \leq b \}$
  \item $[a,b) = \{ x \in \mathbb{R} \mid a \leq x < b \}$
  \end{itemize}
\item \textbf{Intervalos infinitos}:
  \begin{itemize}
  \item $(-\infty,c) = \{ x \in \mathbb{R} \mid x < c \}$
  \item $(c,+\infty) = \{ x \in \mathbb{R} \mid x > c \}$
  \item $(-\infty,+\infty) = \mathbb{R}$
  \end{itemize}
\end{itemize}

Existen combinaciones adicionales de estas definiciones, como por ejemplo $(-\infty,b]$, que se interpretan de manera análoga.
\end{definition}

\begin{example}
Ejemplos de intervalos: $(-1,1)$, $[-2,3)$, $(0,+\infty)$, $\mathbb{R} = (-\infty,+\infty)$.
\end{example}

Las definiciones de intervalos abiertos y cerrados se utilizan para múltiples aspectos de las funciones. Gráficamente se considera el concepto de vecindad, que está representada por un centro y un radio, el cual resulta ser un intervalo: 

\begin{figure}[H]
\centering
\begin{tikzpicture}[scale=1.2]
  % Eje real
  \draw[thick, ->] (-3,0) -- (3,0) node[below right] {$x$};
  
  % Límites de la vecindad (puntos extremos)
  \filldraw (-1.5,0) circle (1.5pt) node[above] {$x_0-\varepsilon$};
  \filldraw (1.5,0) circle (1.5pt) node[above] {$x_0+\varepsilon$};
  
  % Centro x_0
  \filldraw[red] (0,0) circle (2pt) node[below] {$x_0$};
  
  % Línea horizontal superior para representar el intervalo abierto
  \draw[thick, blue] (-1.5,0.3) -- (1.5,0.3);
  
  % Flechas verticales en los extremos para mostrar apertura
  \draw[thick, gray] (-1.5,0) -- (-1.5,0.3);
  \draw[thick, gray] (1.5,0) -- (1.5,0.3);
  
  % Doble flecha para radio \varepsilon (simétrica)
  \draw[thick, |<->|] (0,1) -- (1.5,1) node[above] {$\varepsilon$};
  
  % Etiqueta opcional para el intervalo
  \node[above, blue] at (0,0.3) {$(x_0 - \varepsilon, x_0 + \varepsilon)$};
\end{tikzpicture}
\caption{Intervalo abierto $(x_0 - \varepsilon, x_0 + \varepsilon)$ representado como línea centrada en $x_0$ de radio $\varepsilon$ en el eje real.}
\label{fig:vecindad}
\end{figure}





\begin{definition}[Vecindad]
Dados un número real $x_0 \in \mathbb{R}$ y un número positivo $\varepsilon > 0$, el conjunto $(x_0 - \varepsilon, x_0 + \varepsilon)$ se denomina \textbf{vecindad} de $x_0$ de \textbf{radio} $\varepsilon$, o simplemente vecindad de radio $\varepsilon$ centrada en $x_0$. En este caso, el punto $x_0$ se llama \textbf{centro} de la vecindad, y la magnitud $\dfrac{\varepsilon}{2}$ denomina \textbf{radio de la vecindad}. La Figura \ref{fig:vecindad} representa la vecindad $(x_0 - \varepsilon, x_0 + \varepsilon)$ del punto $x_0$, cuyo radio es $\varepsilon$.
\end{definition}

\subsection{Problemas propuestos}

Siga estos pasos ordenados para resolver los problemas algebraicos. Estos pasos son útiles tanto para su resolución manual como para generar prompts efectivos con inteligencia artificial.

\textbf{PROCESO SISTEMÁTICO DE RESOLUCIÓN}

\begin{enumerate}
\item \textbf{ANÁLISIS INICIAL}
   \begin{itemize}
   \item Lea todo el problema cuidadosamente.
   \item Identifique: ¿Qué datos se dan? ¿Qué se pide?
   \item Clasifique el problema: ecuación, desigualdad, simplificación, o aplicado.
   \end{itemize}

\item \textbf{PLAN DE SOLUCIÓN}
   \begin{itemize}
   \item Liste las técnicas necesarias: factorización, valor absoluto, despeje, etc.
   \item Si es aplicado, asigne variables a las incógnitas.
   \end{itemize}

\item \textbf{EJECUCIÓN ALGEBRAICA}
   \begin{itemize}
   \item Organice la ecuación/desigualdad claramente.
   \item Identifique simplificaciones: fracciones, factorización, radicales.
   \item Realice operaciones paso a paso, explicando cada acción:
     \begin{itemize}
     \item ``Multiplico por $x-2\neq 0$ para eliminar denominador''
     \item ``Factorizo numerador: $x^2-4=(x-2)(x+2)$''
     \end{itemize}
   \end{itemize}

\item \textbf{VALIDACIÓN DE SOLUCIONES}
   \begin{itemize}
   \item Verifique soluciones extraviadas (raíces, valor absoluto).
   \item Determine dominio: denominadores ≠0, radicales ≥0.
   \item Sustituya en la expresión original.
   \end{itemize}

\item \textbf{PRESENTACIÓN FINAL}
   \begin{itemize}
   \item Expresión simplificada o intervalo solución.
   \item Dominio completo.
   \item Contexto físico (si aplica).
   \end{itemize}
\end{enumerate}



\begin{prob}
Encuentre los valores de $x \in \mathbb{R}$ que resuelven las siguientes ecuaciones o desigualdades. Expresa las soluciones en notación de intervalo cuando corresponda y represente gráficamente cada conjunto solución como vecindad abierta en el eje real (incluya la figura con centro y radio $\varepsilon$ apropiados).

\begin{multicols}{2}
\begin{enumerate}

% I. Ecuaciones Racionales y Radicales (5)
\item \(\displaystyle \frac{3x + 1}{5x + 7} = \frac{6x + 11}{10x - 3}\)
\item \(\displaystyle 2 - \frac{1}{x} = 1 + \frac{3}{x}\)
\item \(\displaystyle \frac{2}{x+5} = \frac{3}{2x + 1} - \frac{5}{6x + 3}\)
\item \(\displaystyle \frac{4}{x - 5} - \frac{2}{x + 4} = \frac{3}{2x - 4}\)
\item \(\displaystyle \frac{1}{\sqrt{x}} = 3 - \frac{1 - 3\sqrt{x}}{\sqrt{x}}\)

% II. Ecuaciones Cuadráticas y Polinómicas (5)
\item \(2x^{2} + 7x - 15 = 0\)
\item \((x - 2)(x + 1) - 6 = 0\)
\item \(4x^{2} - 37x^{2} + 75 = 0\)
\item \(x^{2} - 2x - 15 = 0\)
\item \(20x^{2} + 8x^{2} - 55x - 22 = 0\)

% III. Operaciones Algebraicas con Fracciones (5)
\item \(\displaystyle \frac{6xy^{2}}{7} \times \frac{21x^{2}}{y} \div \frac{32xy^{2}}{91}\)
\item \(\displaystyle \frac{12p^{2}q^{2}}{5} \times \frac{15}{4pq} \div 3\)
\item \(\displaystyle \frac{3}{pq} \times \frac{4p}{p+1}\)
\item \(\displaystyle \frac{x^{2} - x - 6}{2xy} \times \frac{2x^{2}y}{x^{2} - 9}\)
\item \(\frac{4}{\sqrt{2} - 1} + \frac{4 + \sqrt{3}}{\sqrt{2} - 3}\)

% IV. Desigualdades Lineales y Polinómicas (5)
\item \(3x - 2 > 12\)
\item \(2x + 5 \leq 8\)
\item \(-2 - 3x \geq 2\)
\item \(3 - 5x < 11\)
\item \(2x + 5 < 6x - 7\)

% V. Ecuaciones y Desigualdades con Valor Absoluto (5)
\item \(|4x - 1| = 7\)
\item \(|2x + 1| + 1 = 15\)
\item \(\left| \frac{2 - 3x}{5} \right| = 2\)
\item \(\left| \frac{2x + 5}{3} \right| < 1\)
\item \(\frac{3}{|5 - 2x|} < 2\)

% VI. Desigualdades Racionales y Polinómicas (5)
\item \(\displaystyle \frac{x^{2}(x + 2)}{x + 2(x + 1)} \leq 0\)
\item \(\displaystyle \frac{(x^{2} + 1)(x - 3)}{x^{2} - 9} \geq 0\)
\item \(\displaystyle \frac{x^{2} - x}{x^{2} + 2x} \leq 0\)
\item \(\displaystyle \frac{(x + 3)(2 - x)}{(x + 4)(x^{2} - 4)} \leq 0\)
\item \(\displaystyle \frac{x - 2}{3x - 10} \geq 0\)

\end{enumerate}
\end{multicols}
\end{prob}



\chapter{Funciones}\label{funciones}

La función es, tal vez, uno de los conceptos más importantes en matemáticas, pues gran parte de los demás conceptos matemáticos se basan en el conocimiento de las funciones y sus propiedades. Aunque existen muchos tipos de funciones —por ejemplo, el área de un círculo que depende de su radio, el costo de un envío que depende de su peso, o la velocidad que depende del tiempo \(t\) transcurrido—, las funciones no siempre dependen de una sola variable. Por ejemplo, el volumen de un cilindro depende tanto de su altura como de su radio. 

En esta sección nos enfocaremos inicialmente en las funciones de una variable real y desarrollaremos con detalle este concepto fundamental.


\section{Propiedades básicas}

\begin{definition}[Función]
Una función \( f \) es una regla de correspondencia que asigna a cada elemento \( x \) de un conjunto denominado \textbf{dominio}, un único valor \( f(x) \) en otro conjunto, llamado \textbf{codominio}. 

El conjunto de todos los valores \( f(x) \) obtenidos se denomina \textbf{rango} o \textbf{imagen} de \( f \). Es común decir que $x$ es la variable independiente y $f(x)$ es la variable dependiente además de usar la notación $y=f(x)$ de manera indistinta.
\end{definition}
\begin{prob} 
Calcule el dominio de las siguientes funciones
\begin{enumerate}
\item $f(x)=x^2-8x+5.$
\item $f(x)=\sqrt{1-x^2}.$
\item $f(x)=\sqrt{3+x}+\sqrt[4]{7-x}.$
\item $f(x)=\dfrac{x+5}{x-3}$
\end{enumerate}
\begin{myproof}
\begin{enumerate}
\item La función es un polinomio de grado $2$. Los polinomios están definidos para todo $x \in \mathbb{R}$, sin restricciones. Por tanto, \(
\operatorname{Dom}(f) = \mathbb{R}.
\)
\item La raíz cuadrada requiere que su argumento sea no negativo, es decir
\[
1 - x^2 \geq 0 \iff x^2 \leq 1 \iff -1 \leq x \leq 1.
\]
No hay otras restricciones. Por tanto, \(
\operatorname{Dom}(f) = [-1, 1].
\)
\item Analizamos cada radical por separado. Para $\sqrt{3+x}$ se cumple que $3 + x \geq 0 \iff x \geq -3$ y para $\sqrt[4]{7-x}$ se tiene que las raíces de índice par requieren radicando $\geq 0$. Así, $7 - x \geq 0 \iff x \leq 7$. Finalmente la suma está definida cuando ambas lo están, por tanto 
\(\operatorname{Dom}(f) = [-3, 7].\)
\item La función está definida para todo número real excepto donde el denominador sea 0. De esta manera, \(\operatorname{Dom}(f) = \mathbb{R}-\lbrace 3 \rbrace.\)
\end{enumerate}
\end{myproof}
\end{prob}


\begin{rem}[Formas de representar una función]
Existen básicamente cuatro formas de representar una función: verbalmente, describiendo en palabras la relación entre variables; numéricamente, mediante una tabla de valores; visualmente, a través de una gráfica; y algebraicamente, por medio de una fórmula explícita. Aclararemos estas formas en el siguiente ejemplo.
\end{rem}

\begin{example}\label{funcioncosto}
Un contenedor rectangular sin tapa tiene un volumen de \(10\,\text{m}^3\). La longitud de su base es dos veces su ancho. El material para la base cuesta \$10 por metro cuadrado y el material para los lados cuesta \$6 por metro cuadrado. Exprese el costo de los materiales como una función del ancho de la base. Dé una representación verbal, numérica, visual y algebraica.

\begin{myproof}
Primero introducimos las variables geométricas del problema. Sea \(x\) el ancho de la base (en metros), \(2x\) la longitud de la base (en metros) y \(h\) la altura del contenedor (en metros). Como el contenedor no tiene tapa, solo hay base y cuatro lados.

La condición de volumen se escribe como
\[
V = (\text{ancho})\cdot(\text{longitud})\cdot(\text{altura}) = x \cdot 2x \cdot h = 2x^{2}h = 10.
\]
Despejamos la altura en función del ancho:
\[
h = \frac{10}{2x^{2}} = \frac{5}{x^{2}}.
\]

Ahora calculamos el área de cada parte para expresar el costo total. 
El área de la base es
\[
A_{\text{base}} = x \cdot 2x = 2x^{2},
\]
y su costo es
\[
C_{\text{base}} = 10 \cdot A_{\text{base}} = 10 \cdot 2x^{2} = 20x^{2}.
\]

El contenedor tiene cuatro lados rectangulares: dos lados de dimensiones \(x \times h\) y dos de dimensiones \(2x \times h\). El área total de los lados es
\[
A_{\text{lados}} = 2(xh) + 2(2xh) = 2xh + 4xh = 6xh.
\]
Sustituimos \(h = \dfrac{5}{x^{2}}\):
\[
A_{\text{lados}} = 6x \cdot \frac{5}{x^{2}} = \frac{30}{x}.
\]
El costo de los lados es entonces
\[
C_{\text{lados}} = 6 \cdot A_{\text{lados}} = 6 \cdot \frac{30}{x} = \frac{180}{x}.
\]

Sumando ambos aportes obtenemos el costo total como función del ancho \(x\):
\[
C(x) = C_{\text{base}} + C_{\text{lados}} = 20x^{2} + \frac{180}{x}.
\]

Podemos ahora presentar las cuatro representaciones de la función costo:

\textbf{Representación verbal:} A cada valor positivo del ancho \(x\) se le asigna el costo total de construir el contenedor, sumando el costo de la base y el de los cuatro lados, donde la longitud es \(2x\) y la altura se ajusta para que el volumen sea \(10\,\text{m}^3\).

\textbf{Representación algebraica:} La función costo en términos del ancho \(x\) está dada por
\[
C(x) = 20x^{2} + \frac{180}{x}, \quad x > 0.
\]

\textbf{Representación numérica:} Podemos elaborar una tabla de valores, eligiendo algunos anchos \(x\) (en metros) y calculando el costo \(C(x)\) (en dólares). Por ejemplo:
\[
\begin{array}{c|c}
x & C(x) \\
\hline
1   & 20(1)^{2} + \dfrac{180}{1}   = 200 \\
1.5 & 20(1.5)^{2} + \dfrac{180}{1.5} = 20\cdot 2.25 + 120 = 165 \\
2   & 20(2)^{2} + \dfrac{180}{2}   = 80 + 90 = 170 \\
2.5 & 20(2.5)^{2} + \dfrac{180}{2.5} = 20\cdot 6.25 + 72 = 197 \\
\end{array}
\]

\textbf{Representación visual:} La representación gráfica consiste en dibujar la curva de la función
\[
y = C(x) = 20x^{2} + \frac{180}{x}, \quad x > 0,
\]
en el plano \(xy\), donde el eje horizontal representa el ancho \(x\) y el eje vertical representa el costo \(C(x)\). Esta gráfica muestra cómo varía el costo total según el ancho de la base.
\begin{figure}[H]
\centering
\begin{tikzpicture}[scale=0.8]
% Definir la función
\begin{axis}[
    width=12cm,
    height=8cm,
    xlabel={$x$ (ancho en metros)},
    ylabel={$C(x)$ (costo en dólares)},
    xmin=0.5, xmax=4,
    ymin=100, ymax=300,
    grid=major,
    axis lines=left,
    xtick={1,1.5,2,2.5,3},
    ytick={150,200,250},
    legend pos=north east,
    domain=0.5:4,
    samples=100,
    thick
]

% Graficar la función C(x)
\addplot[blue, ultra thick] {20*x^2 + 180/x};
\addlegendentry{$C(x)=20x^2+\frac{180}{x}$}

% Marcar los puntos de la tabla numérica
\addplot[red, mark=*, only marks] coordinates {
    (1,200)
    (1.5,165)
    (2,170)
    (2.5,197)
};
\addlegendentry{Puntos de la tabla}

% Etiquetas de los puntos clave
\node[pin=90:{$(1,200)$}] at (axis cs:1,200) {};
\node[pin=120:{$(1.5,165)$}] at (axis cs:1.5,165) {};
\node[pin=60:{$(2,170)$}] at (axis cs:2,170) {};

% Línea vertical en x=0 (asíntota)
\draw[dashed, gray] (axis cs:0.5,100) -- (axis cs:0.5,300) node[above right] {asíntota};

\end{axis}
\end{tikzpicture}
\caption{Figura del ejemplo \ref{funcioncosto}}

\end{figure}
Con esto hemos descrito el costo como función del ancho en sus cuatro formas: verbal, numérica, visual y algebraica.
\end{myproof}
\end{example}
Aunque conocer las cuatro representaciones de una función (verbal, numérica, algebraica y gráfica) es fundamental, en ocasiones una resulta más conveniente que las demás según el contexto. Por ello, es esencial seleccionar la representación más adecuada para cada situación particular. La siguiente es una propiedad clave para identificar gráficas de funciones:

\begin{rem}[Prueba de la recta vertical]

\begin{multicols}{2}
Una curva en el plano $XY$ es la gráfica de una función $y=f(x)$ si y solo si ninguna recta vertical se interseca con la curva en más de un punto. 

Por ejemplo, la circunferencia de radio $1$ centrada en el origen, dada por $x^2 + y^2 = 1$, no representa una función de $x$, ya que una recta vertical como $x=0.5$ la interseca dos veces.

\begin{figure}[H]
\centering
\begin{tikzpicture}[scale=1.5]
\draw[->] (-1.5,0) -- (1.5,0) node[right] {$x$};
\draw[->] (0,-1.5) -- (0,1.5) node[above] {$y$};
\draw[thick] (0,0) circle (1cm);
\draw[red, very thick] (0.5, -1.5) -- (0.5,1.5) node[above right] {Recta vertical};
\filldraw (0.5, 0.866) circle (2pt) node[above right] {};
\filldraw (0.5, -0.866) circle (2pt) node[below right] {};
\end{tikzpicture}
\caption{Circunferencia unitaria fallando la prueba de la recta vertical.}
\label{fig:circulo-vertical}
\end{figure}
\end{multicols}
\end{rem}

\begin{definition}[Funciones crecientes y decrecientes] Una función $f$ se denomina \textbf{creciente} sobre un intervalo $I$ si $f(x_1)<f(x_2)$ siempre que $x_1<x_2$ para todo $x\in I.$ De manera análoga $f$ se denomina \textbf{decreciente} sobre un intervalo $I$ si $f(x_1)>f(x_2)$ siempre que $x_1<x_2\in I.$
\end{definition}

\begin{example} 
Determine los intervalos donde la función $f(x)=x^2$ es creciente o decreciente.
\begin{myproof} Hagamos inicialmente un bosquejo de la función

\begin{figure}[H]
\centering
\begin{tikzpicture}[scale=1.2]
% Ejes
\draw[->] (-1.6,0) -- (1.6,0) node[right] {$x$};
\draw[->] (0,-0.2) -- (0,2.6) node[above] {$y$};

% Malla ligera cerca de cero
\draw[help lines, step=0.5cm, opacity=0.3] (-1.5,-0.1) grid (1.5,2.5);

% Bosquejo de y = x^2 cerca de cero
\draw[thick, blue, domain=-1.5:1.5, samples=100] plot (\x, {\x*\x}) 
    node[above right] {$y = x^2$};

% Puntos clave cerca de cero
\filldraw (0,0) circle (2pt) node[below left] {$O(0,0)$};
\filldraw (1,1) circle (2pt) node[above right] {};
\filldraw (-1,1) circle (2pt) node[above left] {};

\end{tikzpicture}
\caption{Bosquejo de $y=x^2.$}
\label{fig:parabola-cero}
\end{figure}


Visualmente se observa que la función tiene forma de parábola y vértice en $x=0$, sugiriendo comportamiento diferente a cada lado. Analizamos por casos.

\textbf{Caso 1: $x \geq 0$ (intervalo $[0, +\infty)$).} Sean $x_1, x_2 \geq 0$ con $x_1 < x_2$. Entonces $x_2 - x_1 > 0$.

Calculamos:
\begin{align*}
f(x_2) - f(x_1) &= x_2^2 - x_1^2 = (x_2 - x_1)(x_2 + x_1).
\end{align*}
Como $x_2 - x_1 > 0$ y $x_2 + x_1 \geq x_1 + x_1 = 2x_1 \geq 0$, se tiene $f(x_2) - f(x_1) > 0 \iff f(x_2) > f(x_1)$.

Por tanto, $f$ es estrictamente creciente en $[0, +\infty)$.


\textbf{Caso 2: $x \leq 0$ (intervalo $(-\infty, 0]$).} Sean $x_1, x_2 \leq 0$ con $x_1 < x_2 \leq 0$. Entonces $x_2 - x_1 > 0$.

Ahora:
\begin{align*}
f(x_2) - f(x_1) &= x_2^2 - x_1^2 = (x_2 - x_1)(x_2 + x_1).
\end{align*}
Como $x_2 - x_1 > 0$ pero $x_2 + x_1 \leq x_2 + x_2 = 2x_2 \leq 0$, se tiene $f(x_2) - f(x_1) < 0 \iff f(x_2) < f(x_1)$.

Por tanto, $f$ es estrictamente decreciente en $(-\infty, 0]$.

\end{myproof}
\end{example}

El ejemplo anterior se resolverá con mayor facilidad más adelante, luego de estudiar la derivada. Mientras tanto, observe que hay algunas formas adicionales de representar funciones:

\begin{example}[Función definida por partes] 
A veces las funciones se describen mediante diferentes fórmulas o expresiones en distintas partes de sus dominios. Trace la gráfica de la siguiente función definida por partes: 
\[
f(x) = 
\begin{cases}
3-\dfrac{1}{2}x & \text{si } x < 2 \\
2x-5 & \text{si } x \geq 2 \\
\end{cases}
\]
\begin{myproof}
El dominio de la función es $\mathbb{R}$. Graficamos cada pieza por separado: Para $x<2$: $y=3 - \frac{1}{2}x$ es recta decreciente con pendiente $-\frac{1}{2}$, círculo \textbf{abierto} en $x=2$ ($y=2$) y para $x \geq 2$: $y=2x-5$ es recta creciente con pendiente $2$, círculo \textbf{lleno} en $x=2$ ($y=-1$).

\begin{figure}[H]
\centering
\begin{tikzpicture}[scale=1.2]
\draw[->] (0, -2) -- (0, 4) node[right] {$y$};
\draw[->] (-0.5, 0) -- (4.5, 0) node[right] {$x$};
\node at (2, -0.3) {$2$};

% Rama 1: x < 2, 3 - 0.5x (círculo abierto en x=2)
\draw[thick, blue, domain=-0.5:1.99, samples=50] plot (\x, {3 - 0.5*\x});
\fill[white, opacity=1] (2,2) circle (3pt); % Círculo abierto
\draw[blue] (2,2) circle (3pt);

% Rama 2: x >= 2, 2x-5 (círculo lleno en x=2)
\draw[thick, red, domain=2:4.5, samples=50] plot (\x, {2*\x - 5});
\filldraw[red] (2,-1) circle (3pt);

\node[blue] at (1,3.2) {$3 - \frac{1}{2}x$ ($x<2$)};
\node[red] at (3.5,2) {$2x-5$ ($x \geq 2$)};
\end{tikzpicture}
\caption{Gráfica de la función $f(x)$ definida por partes.}
\end{figure}
\end{myproof}
\end{example}

\begin{definition}[Funciones pares e impares]
Una función $f: D \to \mathbb{R}$ se denomina \textbf{par} si satisface la condición de simetría respecto al eje $y$:
\[
f(-x) = f(x) \quad \forall x \in D \cap (-D) \neq \emptyset.
\]
Análogamente, se denomina \textbf{impar} si satisface la condición de simetría respecto al origen:
\[
f(-x) = -f(x) \quad \forall x \in D \cap (-D) \neq \emptyset,
\]
donde $D \cap (-D) \neq \emptyset$ garantiza que el dominio sea simétrico respecto al origen o contenga al menos pares simétricos.
\end{definition}

\begin{example}
La función $f(x) = x^2$ es par, ya que:
\[
f(-x) = (-x)^2 = x^2 = f(x) \quad \forall x \in \mathbb{R}.
\]
Asimismo, la función $f(x) = x^3$ es impar, pues:
\[
f(-x) = (-x)^3 = -x^3 = -f(x) \quad \forall x \in \mathbb{R}.
\]
Estas identidades algebraicas verifican las respectivas simetrías geométricas en sus gráficas.
\end{example}

En la sección subsiguiente se presentará un catálogo más completo de funciones elementales, permitiendo explorar sistemáticamente estas propiedades simétricas y otras características relevantes del análisis funcional.



\section{Funciones elementales fundamentales}

Existe un catálogo básico de funciones elementales fundamentales ---conocidas también como \emph{modelos matemáticos esenciales}---, que incluye la función potencia, la exponencial, la logarítmica, las trigonométricas y las trigonométricas inversas. Estas constituyen la base para la construcción de modelos matemáticos más complejos, y en este capítulo se analizarán detalladamente sus dominios, rangos y gráficas.


\begin{definition}[Función potencia]
Sea $r \in \mathbb{Q}$ un número racional escrito en su forma reducida $r = \frac{p}{q}$, donde $p \in \mathbb{Z}$ es un entero y $q \in \mathbb{N}^+$ es un entero positivo con $\gcd(p,q)=1$. La función potencia $f(x) = x^r = \sqrt[q]{x^p}$ tiene dominio que depende de la paridad del denominador $q$ y la señal de $r$:

\begin{itemize}
\item Si $q$ es \textbf{impar}, la raíz $q$-ésima está definida para todo $x \in \mathbb{R}$, por lo que $\operatorname{Dom}(f) = \mathbb{R}$.
\item Si $q$ es \textbf{par}, la raíz $q$-ésima de números negativos no está definida en reales, requiriendo $x^p \geq 0$. Para $p$ impar esto equivale a $x \geq 0$, dando $\operatorname{Dom}(f) = [0, +\infty)$.
\item Si $r < 0$, la expresión involucra división por $x^{-r}$, requiriendo excluir $x=0$: $\operatorname{Dom}(f) = \mathbb{R} \setminus \{0\}$ (o $(0,+\infty)$ si $q$ par).
\end{itemize}

Como casos particulares, para enteros positivos $n \in \mathbb{N}$, $f(x) = x^n$ es un polinomio definido en todo $\mathbb{R}$. Para enteros negativos $n < 0$, $f(x) = x^n = \frac{1}{x^{-n}}$ está definida en $\mathbb{R} \setminus \{0\}$.

Estas reglas garantizan que $x^r$ produzca valores reales para todo $x$ en su dominio respectivo.
\end{definition}





\begin{example}[Ejemplos de funciones potencia elementales] Listamos algunos ejemplos representativos
\begin{figure}[h]
\centering
\begin{tikzpicture}[scale=0.8]
\draw[->] (-2,0) -- (2,0) node[right] {$x$};
\draw[->] (0,-1.5) -- (0,2) node[above] {$y$};
\draw[thick, domain=-1.8:1.8, samples=100] plot (\x, {\x});
\node[above right] at (1.5,1.5) {$f(x)=x$};
\end{tikzpicture}
\caption{$f(x)=x$: $\operatorname{Dom} = \mathbb{R}$.}
\end{figure}


\begin{figure}[H]
\centering
\begin{tikzpicture}[scale=0.8]
\draw[->] (-1.5,0) -- (1.5,0) node[right] {$x$};
\draw[->] (0,-0.2) -- (0,1.8) node[above] {$y$};
\draw[thick, domain=-1.4:1.4, samples=100] plot (\x, {\x*\x});
\node[above right] at (1,1) {$f(x)=x^2$};
\end{tikzpicture}
\caption{Función cuadrática $f(x)=x^2$: $\operatorname{Dom} = \mathbb{R}$.}
\end{figure}

\begin{figure}[H]
\centering
\begin{tikzpicture}[scale=0.8]
\draw[->] (-1.5,0) -- (1.5,0) node[right] {$x$};
\draw[->] (0,-1.2) -- (0,1.8) node[above] {$y$};
\draw[thick, domain=-1.4:1.4, samples=100] plot (\x, {\x*\x*\x});
\node[above right] at (1,1) {$f(x)=x^3$};
\end{tikzpicture}
\caption{Función cúbica $f(x)=x^3$: $\operatorname{Dom} = \mathbb{R}$.}
\end{figure}

\begin{figure}[H]
\centering
\begin{tikzpicture}[scale=0.8]
\draw[->] (-1.5,0) -- (1.5,0) node[right] {$x$};
\draw[->] (0,-0.2) -- (0,2) node[above] {$y$};
\draw[thick, domain=-1.4:1.4, samples=100] plot (\x, {\x*\x*\x*\x});
\node[above right] at (1,1) {$f(x)=x^4$};
\end{tikzpicture}
\caption{Función $f(x)=x^4$: $\operatorname{Dom} = \mathbb{R}$, par.}
\end{figure}

\begin{figure}[H]
\centering
\begin{tikzpicture}[scale=0.8]
\draw[->] (-2,-2) -- (2,-2) node[right] {$x$};
\draw[->] (0,-2) -- (0,3) node[above] {$y$};
\draw[thick, blue, domain=-1.8:-0.1, samples=100] plot (\x, {1/\x});
\draw[thick, blue, domain=0.1:1.8, samples=100] plot (\x, {1/\x});
\node[blue, above right] at (1.2,1) {$f(x)=1/x$};
\end{tikzpicture}
\caption{Función recíproca $f(x)=1/x$: $\operatorname{Dom} = \mathbb{R} \setminus \{0\}$}
\end{figure}



\begin{figure}[H]
\centering
\begin{tikzpicture}[scale=0.8]
\draw[->] (-2,0) -- (2,0) node[right] {$x$};
\draw[->] (0,-0.2) -- (0,1.6) node[above] {$y$};
\draw[thick, domain=-1.8:1.8, samples=200] plot (\x, {abs(\x)^(2/3)});
\filldraw (0,0) circle (2pt) node[below] {$O$};
\node[above right] at (1.2,1) {$f(x)=x^{2/3}$};
\end{tikzpicture}
\caption{Función $x^{2/3}$: $\operatorname{Dom} = \mathbb{R}$.}
\end{figure}

\end{example}












\chapter{Técnicas de integración}

\section{Ejercicios}

\begin{prob} Calcule las siguientes integrales indefinidas
\begin{multicols}{2}
\begin{enumerate}
\item $\displaystyle\int \sqrt{1-4y}\ dy.$
\item $\displaystyle\int x^4\sqrt{3x^5-5}\ dx.$
\item $\displaystyle\int y^2\sec^2(y^3)\ dy.$
\item $\displaystyle\int \dfrac{\left( x^2-4x+6\right)}{x^3-6x^2+18x}\ dx.$
\item $\displaystyle\int x^2 e^x \cos x\,dx.$
\item $\displaystyle\int \dfrac{x^2+2x}{\sqrt{x^3+3x^2+1}}\ dx.$
\item $\displaystyle\int \sec x\tan x\cos\left( \sec x \right)\ dx.$
\item $\displaystyle\int x^2e^x\ dx.$
\item $\displaystyle\int \cos^{2} x\ dx.$
\item $\displaystyle\int \dfrac{xe^x}{\left( x+1\right)^2}\ dx.$
\item $\displaystyle\int \sen t\ln\left( \cos t \right)\ dt.$
\item $\displaystyle\int x\sec^2 x\ dx.$
\item $\displaystyle\int x\arctan x\ dx.$
\item $\displaystyle\int e^x\cos x\ dx.$
\item $\displaystyle\int \sqrt{20-8x-x^2}\ dx.$
\item $\displaystyle\int \dfrac{5x-1}{x^2-1}\ dx.$
\item $\displaystyle\int \dfrac{3x-13}{(x+2)(x-5)}\ dx.$
\item $\displaystyle\int \cos x\sqrt{4-\sen^2 x}\ dx.$
\item $\displaystyle\int \dfrac{2x^2+3x+2}{x^3+4x^2+6x+4}\ dx.$
\item $\displaystyle\int \dfrac{1}{9x^4+x^2}\ dx.$
\item $\displaystyle\int \dfrac{e^{5x}}{\left(e^{2x} + 1 \right)^2}\ dx.$
\item $\displaystyle\int \dfrac{\ln x}{x^2}\ dx.$
\item $\displaystyle\int \dfrac{2x^2-x-20}{x^2+x-6}\ dx.$
\item $\displaystyle\int \dfrac{2x^2-5x+1}{x^3-2x^2-x+2}\ dx.$
\item $\displaystyle\int \frac{x^3+2x^2-3x+1}{x^2-1}\,dx.$
\item $\displaystyle\int \frac{x^2}{(x-1)^2(x^2+4)}\,dx.$
\item $\displaystyle\int \frac{2x^3-5x^2+4x-1}{(x^2+1)^2}\,dx.$
\item $\displaystyle\int \frac{x^2\ln x}{x^2+1}\,dx.$
\end{enumerate}
\end{multicols}
\end{prob}

%\chapter*{Bibliografía}

\nocite{*} % Incluye todos los items del archivo BibTeX
\bibliographystyle{apalike}
\bibliography{anfalearNotasCalculo}
\addcontentsline{toc}{chapter}{Bibliografía}

\end{document}

