\chapter{Políticas del curso}

\noindent\fbox{\parbox{\textwidth}{\textit{
La información aquí presentada no se modificará por ninguna de las partes a menos que haya un consenso entre estas, el cual debe ser oportunamente informado y reportado por escrito en este documento. Ante cualquier instancia, este documento será soporte para las acciones a las que se de lugar, por lo cual, tanto el profesor como el estudiante lo aceptan y se someten a lo que esté aquí escrito. 
}}}

Esta política se rige de acuerdo a lo dispuesto en los Reglamentos Académicos de Pregrado de cada institución (R.A.P.) disponibles en las web institucionales.

Los cursos de Álgebra Lineal tendrán su Aula Virtual, donde se informarán horarios de atención, distribución de exámenes, demás aspectos particulares, calificaciones y distribuciones de porcentajes. La nota mínima de aprobación es 3.0.

\section{Contenido del curso} 

La cobertura total del contenido dependerá del progreso del semestre; es responsabilidad del estudiante mantenerse al día. Los temas evaluados en cada corte se proporcionarán oportunamente. Los temas marcados con asterisco (*) son opcionales según el progreso del semestre.

\begin{enumerate}
\item \textbf{Funciones}
\begin{enumerate}[$a)$]
\item Concepto y representación de funciones.
\item Clasificación y gráficas de funciones:
    \begin{itemize}
        \item Algebraicas: polinómicas, racionales e irracionales.
        \item Trascendentes: exponenciales, logarítmicas, trigonométricas, trigonométricas inversas e hiperbólicas.
        \item Otras funciones: valor absoluto y parte entera.
    \end{itemize}
\item Operaciones con funciones:
    \begin{itemize}
        \item Álgebra de funciones.
        \item Composición de funciones.
        \item Función inversa.
    \end{itemize}
\item Representación de funciones como modelos matemáticos.
\end{enumerate}

\item \textbf{Límites y continuidad de funciones}
\begin{enumerate}[$a)$]
\item Concepto de límite de una función.
\item Límites laterales, infinitos y especiales.
\item Teoremas de límites y cálculo de límites utilizando teoremas.
\item Límites de funciones trigonométricas.
\item Continuidad de funciones.
\end{enumerate}

\item \textbf{Derivada}
\begin{enumerate}[$a)$]
\item Interpretación geométrica de la derivada.
\item Velocidad promedio y velocidad instantánea.
\item Concepto y reglas para el cálculo de la derivada (incluyendo la regla de la cadena).
\item Derivadas de:
    \begin{itemize}
        \item Funciones algebraicas (teoremas y cálculo).
        \item Funciones trigonométricas e inversas.
        \item Funciones exponenciales y logarítmicas.
        \item Funciones hiperbólicas.
    \end{itemize}
\item Derivadas de orden superior.
\item Derivación implícita y función inversa.
\end{enumerate}

\item \textbf{Aplicaciones de la derivada}
\begin{enumerate}[$a)$]
\item Tasas de cambio relacionadas.
\item Diferenciales y aproximaciones.
\item Regla de L’hôpital.
\item Trazado de gráficas:
    \begin{itemize}
        \item Máximos y mínimos.
        \item Crecimiento, decrecimiento y concavidades.
        \item Puntos de inflexión y asíntotas.
    \end{itemize}
\item Teorema del valor medio.
\item Problemas de optimización.
\end{enumerate}

\item \textbf{Introducción a las integrales}
\begin{enumerate}[$a)$]
\item Concepto de antiderivadas o integración indefinida.
\item Técnicas de integración:
    \begin{itemize}
        \item Sustitución simple.
        \item Por partes.
        \item De funciones racionales mediante completación de cuadrados o fracciones parciales.
        \item Con productos y potencias de funciones trigonométricas.
        \item Sustitución trigonométrica.
    \end{itemize}
\item Integrales impropias.
\end{enumerate}

\item \textbf{La integral definida}
\begin{enumerate}[$a)$]
\item Concepto y propiedades de la integral definida.
\item El teorema fundamental del cálculo.
\item Cálculo de integrales definidas.
\item Integración numérica.
\end{enumerate}

\item \textbf{Aplicaciones de la integral}
\begin{enumerate}[$a)$]
\item Cálculo de áreas de regiones planas.
\item Volumen de sólidos de revolución (capas, discos, arandelas y cascarones).
\item Longitud de arco.
\item Áreas de superficies de revolución.
\item Centro de masa, centro de inercia y trabajo.
\end{enumerate}

\item \textbf{Ecuaciones paramétricas y coordenadas polares}
\begin{enumerate}[$a)$]
\item Representación paramétrica de curvas en el plano.
\item Sistema de coordenadas polares.
\item Gráfica de ecuaciones polares.
\item Cálculo de áreas y longitud en coordenadas polares.
\end{enumerate}
\end{enumerate}


\textbf{Notas aclaratorias:}
\begin{itemize}
\item No hay número específico de quices o trabajos; si son varios se promedian.
\item Las calificaciones se publican en la plataforma universitaria con notificación por correo. Si el estudiante no puede visualizar su nota, debe manifestarlo por correo al profesor.
\item El profesor acordará con los estudiantes la forma de revisar trabajos y exámenes.
\end{itemize}

\section{Políticas de clase}

\begin{itemize}
\item Los instrumentos electrónicos requieren aprobación del profesor.

\item Quices y exámenes sin apuntes, libros ni aparatos electrónicos (salvo flexibilización informada). El estudiante debe leer las reglas de cada evaluación.

\item El trabajo debe ser autónomo o en grupo designado. \textbf{Ofrecer} o \textbf{aceptar} soluciones ajenas constituye \textbf{plagio}, penalizado según el R.A.P. Incluye omisión de fuentes e uso no citado de IA.

\item \textbf{No se reciben trabajos fuera de fecha}, excepto con excusa según R.A.P.

\item \textbf{Asistencia obligatoria} con control por parte del profesor.

\item En exámenes escritos se esperan 15 minutos tras el inicio para ingresar; después se requiere supletorio según R.A.P.

\item \textbf{Casos de emergencia:} El profesor informará evacuaciones necesarias. No se responsabiliza por decisiones estudiantiles post-evacuación.
\end{itemize}